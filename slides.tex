\documentclass[handout,t,aspectratio=1610]{beamer}
\makeatletter
\setbeamertemplate{algorithm}[numbered]
\makeatother

% ======================= Basics =======================
\usepackage{pgfpages}
\usepackage{xcolor}
\usepackage{natbib}


\usepackage{graphicx}
\usepackage{epsfig}
\usepackage{bbm}
\usepackage{calc}
\usepackage{capt-of}
\usepackage{geometry}
\usepackage{blkarray}
\usepackage{mathtools}

\usepackage{amsmath}
\usepackage{algorithm}
\usepackage{algorithmic}
\usepackage{algpseudocode}

% ======================= Theme =======================
\usetheme{Berlin}
\usecolortheme{dolphin}
\useoutertheme{default}
\setbeamertemplate{blocks}[rounded][shadow=false]




% ======================= Color & highlight =======================
\definecolor{lightpurple}{RGB}{242, 232, 255}
\definecolor{lightblue}{RGB}{232, 240, 255}

% For highlighting text
\usepackage{soul}
\usepackage{tcolorbox}
\newcommand{\hlp}[1]{\colorbox{lightpurple}{#1}}
\newcommand{\hlb}[1]{\colorbox{lightblue}{#1}}

\newtcolorbox{pbox}{
    enhanced,
    colback=lightpurple,
    boxrule=0pt,
    sharp corners,
    break=true,
    frame empty,
    left=2pt,
    right=2pt,
    top=2pt,
    bottom=2pt,
    parbox=false
}

\newtcolorbox{bluebox}{
    enhanced,
    colback=lightblue,
    boxrule=0pt,
    sharp corners,
    break=true,
    frame empty,
    left=2pt,
    right=2pt,
    top=2pt,
    bottom=2pt,
    parbox=false
}

% For figures and caption
\usepackage{caption}
\usepackage{subcaption}

% For slides settings
\beamerdefaultoverlayspecification{<+->}
\setbeamertemplate{subtitle}{}
\setbeamertemplate{footline}{
    \hfill\insertframenumber/\inserttotalframenumber
    \hspace{1em}\vspace{1ex}
}

\AtBeginSection[]
{
    \begin{frame}
        \frametitle{Outline}
        \tableofcontents[currentsection]
    \end{frame}
}


% #############################################################################
% #############################################################################
\title{Title}
\author{Mellen Y. Pu}
\date{\today}
\titlegraphic{\includegraphics[width=4cm]{./common/logo}}
% #############################################################################
% #############################################################################
\begin{document}

\frame{\titlepage}

\section[Outline]{}
\begin{frame}{Outline}
    \tableofcontents
\end{frame}


% -----------------------------------------------------------------------------
\section{Introduction}

    % Single Image Frame
    \begin{frame}{Single Image Example}
        \begin{center}
            \includegraphics[height=0.8\textheight]{example-image-a}

            \vspace{0.5em}
            {\small A high-resolution microscopy image showing cellular structure}
        \end{center}

        % Optional: Add brief explanation below
        \vspace{0.5em}
        \begin{itemize}
            \item Key features are marked in different colors
            \item Scale bar: 10 cm
        \end{itemize}
    \end{frame}

    % Version with more vertical space for image
    \begin{frame}{}
        \vspace*{0.15cm}  % Adjust vertical space if needed
        \begin{center}
            \includegraphics[height=0.95\textheight]{example-image-a}
        \end{center}
    \end{frame}

    % Side-by-Side with Annotations
    \begin{frame}{Comparative Analysis}
        \begin{columns}[T]
            \begin{column}{0.48\textwidth}
                \centering
                \includegraphics[width=\textwidth]{example-image-a}

                \vspace{0.3em}
                {\small Control group}

                \begin{itemize}
                    \item Base measurement
                    \item Standard conditions
                \end{itemize}
            \end{column}
            \begin{column}{0.48\textwidth}
                \centering
                \includegraphics[width=\textwidth]{example-image-b}

                \vspace{0.3em}
                {\small Treatment group}

                \begin{itemize}
                    \item 35\% improvement
                    \item Modified parameters
                \end{itemize}
            \end{column}
        \end{columns}
    \end{frame}

    \begin{frame}{Background}
        Recent studies ... The methodology proposed by \cite{zhang2024} demonstrates improved efficiency.
    \end{frame}

    \begin{frame}{Challenges}
        Some thing here.
    \end{frame}


    \begin{frame}{Our Approach}
        \begin{block}{Solution}
            \begin{enumerate}
                % Use this command if in previous page you counted something.
                % \setcounter{enumi}{7}
                \item
            \end{enumerate}
        \end{block}
    \end{frame}


    \begin{frame}{Contribution}
        \begin{block}{}
            \small
            \begin{enumerate}
                \item Summarize the contribution.
            \end{enumerate}
        \end{block}
    \end{frame}


% -----------------------------------------------------------------------------
\section{Methodology}
\begin{frame}{Framework}


\end{frame}

\begin{frame}[allowframebreaks]{Algorithm Template}
   \begin{algorithm}[H]
       \caption{Simple Template}
       \begin{algorithmic}[1]
            \FOR{i = 1 \TO n}
                \STATE sum $\gets$ sum + i
                \IF{sum $\nabla $ threshold}
                    \STATE flag $\gets$ true    \textcolor{red}{\triangleright Comments. }
                \ELSIF{sum = threshold}
                    \STATE count $\gets$ count + 1
                \ELSE
                    \STATE flag $\gets$ false
                \ENDIF
            \ENDFOR
            \WHILE{x $<$ 10}
                \STATE x $\gets$ x + 1
            \ENDWHILE
            \RETURN result
       \end{algorithmic}
   \end{algorithm}
\end{frame}


% Alternative frame with dcases for equation cases
\begin{frame}{Equation Cases with Explanations}
    \[
        |x| = \begin{dcases}
            \underbrace{x}_{\text{when x = 0}}
        \end{dcases}
    \]

    % Function definition with cases
    \[
        f(n) = \begin{dcases}
            \underbrace{n/2}_{\text{even case}} & \text{if } n \text{ is even} \\[1ex]
            \underbrace{3n + 1}_{\text{odd case}} & \text{if } n \text{ is odd}
        \end{dcases}
    \]
\end{frame}

\begin{frame}{Formula Explanation with Braces}
    Example 1: Simple underbrace:
    \[
        f(x) = \underbrace{-\frac{1}{16}x^2}_{
            \substack{\text{quadratic} \\ \text{term}}
        } + \underbrace{4x}_{
            \text{linear term}
        } + \underbrace{7}_{\text{constant}}
    \]

    Example 2: Multiple levels of braces:
    \[
        E = \underbrace{
            \underbrace{mc^2}_{\text{rest energy}} +
            \underbrace{\frac{1}{2}mv^2}_{\text{kinetic energy}}
        }_{\text{total energy}}
    \]

    Example 3: Combining over and under braces:
    \[
        \overbrace{
            a + b + c
        }^{\text{sum}} =
        \underbrace{
            \frac{(a+b+c)}{3}
        }_{\text{average}} \times 3
    \]

\end{frame}


\begin{frame}{Text Highlighting Examples}
    \begin{pbox}
    This is an example of a multi-line highlighted text block.
    The highlighting continues seamlessly across line breaks,
    making it perfect for emphasizing entire paragraphs or
    long sections of text.
    \end{pbox}

    \vspace{1em}

    Regular text with \hlp{highlighted words} in between. You can
    highlight \hlb{specific terms} or \hlp{important concepts}
    within a sentence. The highlighting can be used for
    \hlb{individual words} or \hlp{short phrases}.

    \vspace{1em}

    \begin{bluebox}
    Another example of multi-line highlighting, this time in
    light blue. This approach is particularly useful when you
    need to emphasize large blocks of text while maintaining
    readability.
    \end{bluebox}
\end{frame}


\begin{frame}{Mix Different Highlights}

    Mix different highlights in the same paragraph: Here's some
    text with \hlp{purple highlights} mixed with \hlb{blue highlights}
    to emphasize different \hlp{concepts} or \hlb{ideas}.

    % Example with justified text
    \begin{bluebox}
    \justifying
    This is an example of justified text within a highlighted box.
    The text will stretch to fill the width of the frame while
    maintaining the highlight effect. This is particularly useful
    for formal presentations where text alignment is important.
    \end{bluebox}

\end{frame}

% Alternative frame showing different styles
\begin{frame}{Advanced Highlighting Techniques}
    % Custom width highlight box
    \begin{tcolorbox}[
        colback=lightpurple,
        boxrule=0pt,
        width=0.99\textwidth,
        left=2pt,
        right=2pt
    ]
    A narrower highlighted block that doesn't span
    the full width of the frame.
    \end{tcolorbox}

    \vspace{1em}

    % Inline mixed highlighting
    This paragraph demonstrates how to \hlp{highlight specific terms}
    while keeping the rest of the text normal. You can even
    \hlb{nest different \hlp{highlight} colors} for emphasis.

    \vspace{1em}

    % List with highlights
    \begin{itemize}
        \item Point one with \hlp{highlighted terms}
        \item \hlb{Entire point highlighted in blue}
        \item Point three with \hlp{purple} and \hlb{blue} highlights
    \end{itemize}

    % Quote with highlight
    \begin{bluebox}
    \begin{quote}
    A quoted text block with highlighting. Perfect for emphasizing
    important quotes or references in your presentation.
    \end{quote}
    \end{bluebox}
\end{frame}


% -----------------------------------------------------------------------------
\section{Experiments}
\subsection{Main experiment (1 page in total)}
\begin{frame}{Experiments}
    \begin{enumerate}
        \item Description of the main experiments.
    \end{enumerate}

\end{frame}



\subsection{Ablation study (3 in total)}
\begin{frame}{Part 1}
    \begin{columns}[T]
       \begin{column}{0.5\textwidth}
           \begin{block}{Part 1}
               \begin{enumerate}
                   \item Items to be tested.
               \end{enumerate}
           \end{block}
       \end{column}

        \hfill

       \begin{column}{0.5\textwidth}
           \begin{center}
               \includegraphics[width=\textwidth,height=0.7\textheight]{example-image}
           \end{center}
       \end{column}
    \end{columns}

\end{frame}


\begin{frame}{Part 2}
    \begin{columns}[T]
       \begin{column}{0.5\textwidth}
           \begin{block}{Embedding methods}
               \begin{enumerate}
                    \item Items.
               \end{enumerate}
           \end{block}
       \end{column}

        \hfill

       \begin{column}{0.5\textwidth}
           \begin{center}
               \includegraphics[width=\textwidth,height=0.7\textheight]{example-image}
           \end{center}
       \end{column}
    \end{columns}

\end{frame}



\begin{frame}{Part 3}
    Some other ablation experiments.
\end{frame}




% -----------------------------------------------------------------------------
\section{Related Work}
\subsection{Sub-title for another topic}
\begin{frame}{Part 1}
Formal academic review.

\end{frame}



\subsection{Sub-title for another topic}
\begin{frame}{Part 2}


\end{frame}


% -----------------------------------------------------------------------------
\section{Conclusion}
\begin{frame}{Conclusion}


\end{frame}

\section{Reference}
\begin{frame}[allowframebreaks]{References}

\bibliography{slides}
\bibliographystyle{apalike}

\end{frame}

\end{document}
